% T I T L E   P A G E
% -------------------
% Last updated Nov 1, 2016, by Stephen Carr, IST-Client Services
% The title page is counted as page `i' but we need to suppress the
% page number.  We also don't want any headers or footers.
\pagestyle{empty}
\pagenumbering{roman}

% The contents of the title page are specified in the "titlepage"
% environment.
\begin{titlepage}
        \begin{center}
        \vspace*{1.0cm}

        \Huge
        {\bf Efficient Structure-aware OLAP Query Processing over Large Property Graphs}

        \vspace*{1.0cm}

        \normalsize
        by \\

        \vspace*{1.0cm}

        \Large
        Yan Zhang \\

        \vspace*{3.0cm}

        \normalsize
        A thesis \\
        presented to the University of Waterloo \\ 
        in fulfillment of the \\
        thesis requirement for the degree of \\
        Master of Mathematics \\
        in \\
        Computer Science \\

        \vspace*{2.0cm}

        Waterloo, Ontario, Canada, 2017 \\

        \vspace*{1.0cm}

        \copyright\ Yan Zhang 2017 \\
        \end{center}
\end{titlepage}

% The rest of the front pages should contain no headers and be numbered using Roman numerals starting with `ii'
\pagestyle{plain}
\setcounter{page}{2}

\cleardoublepage % Ends the current page and causes all figures and tables that have so far appeared in the input to be printed.
% In a two-sided printing style, it also makes the next page a right-hand (odd-numbered) page, producing a blank page if necessary.
 


% D E C L A R A T I O N   P A G E
% -------------------------------
  % The following is a sample Delaration Page as provided by the GSO
  % December 13th, 2006.  It is designed for an electronic thesis.
  \noindent
I hereby declare that I am the sole author of this thesis. This is a true copy of the thesis, including any required final revisions, as accepted by my examiners.

  \bigskip
  
  \noindent
I understand that my thesis may be made electronically available to the public.

\cleardoublepage

% A B S T R A C T
% ---------------

\begin{center}\textbf{Abstract}\end{center}


Property graph model is a semantically rich model for real-world applications that represent their data as graphs, e.g., communication networks, social networks, financial transaction networks and etc. On-Line Analytical Processing (OLAP) provides an important tool for data analysis by allowing users to perform data aggregation through different combinations of dimensions. For example, given a Q\&A forum dataset, in order to study if there is a correlation between a poster's age and his or her post quality, one may ask what is the average user's age grouped by the post score. Another example is that, in the field of music industry, it may be interesting to ask what total sales of records is with respect to different music companies and years so as to conduct a market activity analysis.


Surprisingly, current graph databases do not efficiently support OLAP aggregation queries. On the contrary, in most cases they transfer such queries into a sequence of operations and compute everything from scratch. For example, Neo4j, a state-of-art graph database system, processes each OLAP query in two steps. First, it expands the nodes and edges that satisfy the given query constraint. Then it performs the aggregation over all the valid substructures returned from the first step. However, in warehousing data analysis workloads, it is common to have repeating queries from time to time. Computing everything from scratch would be highly inefficient. Moreover, since most graph database systems are disk-based due to the large size of real-world property graphs, it is infeasible to directly employ a graph database system like Neo4j for such OLAP workloads. %It is unacceptable for users to wait for hours for a single query to return. 

Materialization and view maintenance techniques developed in traditional RDBMS have proved to be efficient and critical for processing OLAP workloads. Following the generic materialization methodology, in this thesis we develop a structure-aware cuboid caching solution to efficiently support OLAP aggregation queries over property graphs. The essential idea is to precompute and materialize some views wisely based on history workload, such that future workload processing can be accelerated. %We implement a prototype system upon Neo4j that greatly improves efficiency of OLAP over large property graphs.   

We implemented a prototype system on top of Neo4j. Empirical studies over real-world property graph in different size scales show that, with a reasonable space cost constraint, our solution usually achieves 10-30x speedup over native Neo4j in time efficiency.



\cleardoublepage

% A C K N O W L E D G E M E N T S
% -------------------------------

\begin{center}\textbf{Acknowledgements}\end{center}

I would like to thank Professor M. Tamer {$\ddot{\mbox{O}}$}zsu and Dr. Xiaofei Zhang who made this thesis possible.
\cleardoublepage

% D E D I C A T I O N
% -------------------

\begin{center}\textbf{Dedication}\end{center}

This is dedicated to my mother Limei Leng whom I love.
\cleardoublepage

% T A B L E   O F   C O N T E N T S
% ---------------------------------
\renewcommand\contentsname{Table of Contents}
\tableofcontents
\cleardoublepage
\phantomsection    % allows hyperref to link to the correct page

% L I S T   O F   T A B L E S
% ---------------------------
\addcontentsline{toc}{chapter}{List of Tables}
\listoftables
\cleardoublepage
\phantomsection		% allows hyperref to link to the correct page

% L I S T   O F   F I G U R E S
% -----------------------------
\addcontentsline{toc}{chapter}{List of Figures}
\listoffigures
\cleardoublepage
\phantomsection		% allows hyperref to link to the correct page

% GLOSSARIES (Lists of definitions, abbreviations, symbols, etc. provided by the glossaries-extra package)
% -----------------------------
\printglossaries
\cleardoublepage
\phantomsection		% allows hyperref to link to the correct page

% Change page numbering back to Arabic numerals
\pagenumbering{arabic}

