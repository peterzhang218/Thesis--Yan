
%----------------------------------------------------------------------
\section{Preliminaries}
%----------------------------------------------------------------------
\subsection{Notation of Query}


In this thesis, we use 

\textit{Structure : Dimension, Measure, Slicing Condition}

to represent an OLAP query. 

For instance Query \#3 

Get average user age grouped by users’ 2017 posts’ tags. 

is written as

User-Post, Post-Tag: Tag.Tagname, AVG(User.Age), Post.Year=2017

For a query q, we use q.properties to refer to a set of all properties in Dimension, Measure, and Slicing Condition of q. Suppose q is Query \#3, then

 q.properties=\{Tag.Tagname, User.Age, Post.Year\}.
 
 Similarly,
 
 q.structure= User-Post, Post-Tag.


\subsection{Cuboid vs Substructures}
Suppose we have the following 7 queries in previous workload:

User-Post, Post-Tag: (AVG)User.Age, Tag.TagName

User-Post, Post-Tag: User.CreationDate\_Year, Tag.TagName

User-Post, Post-Tag: User.CreationDate\_Year, (AVG)Post.Score,Tag.TagName=database

User-Post, Post-Tag: User.Age, (AVG)Post.Score,Tag.TagName=java

User-Post, Post-Tag: User.UpVotes, (AVG)Post.Score,Tag.TagName=algorithm

Badge-User, User-Post, Post-Tag: Tag.TagName, Badge.Name=Student, Post.CreationDate\_Year

Badge-User, User-Post, Post-Tag: Tag.TagName, Badge.Name=Teacher

We can tell that the user is interested in “User-Post, Post-Tag” structure. Suppose we build a cube as in Graph Cube [4] on “User-Post, Post-Tag” over all queried attributes and precompute some most beneficial cuboids. 

For instance we precompute cuboids like

User-Post, Post-Tag: User.Age, User.UpVotes, Tag.TagName, User.CreationDate\_Year, Post.Score
Cuboids only contain attributes. 

They do not contain IDs of nodes. Cuboids can only be used in queries with exactly the same structure. They can be be scanned for more aggregated dimension combinations(drill-down operations). Notice that cuboids are not useful for queries with different structures.

We cannot process

Badge-User, User-Post, Post-Tag: Tag.TagName, Badge.Name=Teacher

by simply  joining cuiboid

User-Post, Post-Tag: User.Age, User.UpVotes, Tag.TagName, User.CreationDate\_Year, Post.Score

because IDs of User are not provided in the cuboid.

If besides properties, we also keep IDs of User, then the materilization is joinable on User. The drawback is that result would be much more space-costly than cuboids, as ID of User is unique key. 

We call materializations that only aggregates on properties without storing IDs “cuboids”, and we call those with IDs as “substructures”. The trade-off between “cuboids” and “substructures” is space vs usage potential.

Cuboids are generally ligher in terms of space cost, but they can help with queries with exactly same structure.

Substructures generally heavier in terms of space cost, but it can be widely used to join with other substructures.




%----------------------------------------------------------------------
\section{Problem Definition}
%----------------------------------------------------------------------
Given previous queires P, space limit $\sigma$, which cuboids C and substructures S shall we materialize so that future queries F could be faster processed using C and S? 

Given P, space limit $\sigma$, select C and S wisely.

Given F, C and S, process F fast.

