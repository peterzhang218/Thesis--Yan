Our work addresses on the urgent need for efficient OLAP processing over large property graphs.  To model the problem mathematically, we provided formal definitions of ``Materialization Selection'', ``Execution Planning'', ``Decomposition Problem'' and ``Composition Problem''. We proposed an end-to-end system to tackle these problems and implemented the system for Neo4j. The main idea of our solution is to accelerate future query processing with materialized views which were selected based on their benefits to previous workload. We conducted experiments on our system and it was proved to be a feasible solution that achieves efficient OLAP queries processing on large graph datasets.   

We summarize future work as follows. First, as reflected in experiment results \ref{Results and Discussion}, efficiency performance of our system is affected by system settings, hence study on appropriate system settings would be a valuable follow-up for our work. Second, the system we implemented so far supports SPARQL like queries over schema graph, instead of data graph. It is realizable to enable processing SPARQL like queries over data graph within our proposed solution framework. The key issue is take isomorphism into consideration during query decomposition and composition. Last but not least, greedy selection framework is with no doubt not a perfect solution for ``Materialization Selection'' problem. We look forward to better solutions for this hard but valuable problem.


