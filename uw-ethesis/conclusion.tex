Our work addresses the urgent need for efficient OLAP processing over large property graphs.  We present an end-to-end system to tackle the problem that we integrate into Neo4j. The main idea of our solution is to accelerate future query processing using materialized views that were selected based on their benefits in previous workload. We emphasize structure as a key feature of graph OLAP queries, which is a complement of previous work \cite{sigmod11_ZhaoLXH11}. We introduce and distinguish the concepts of cuboid and substructure, which are two important types of materialized views and should be handled differently. Our solution aims at property-aware and structure-aware materialization by taking both cuboids and substructures into account. To achieve this, a greedy selection framework is proposed for materialized view selection. Besides, we provide three approaches for substructure join and data retrieval from Neo4j, and analyze their advantages and disadvantages. Our solution works for any SPARQL-like query over the schema graph of any property graph, and it is compatible for other graph databases. We conduct experiments by running a set of queries using our system and native Neo4j. Results show that with acceptable space cost in materialization (12.8\% of the total size of the dataset), our system achieves 30x speedup over native Neo4j. The experiments demonstrate our system to be a feasible solution that achieves efficient OLAP queries processing on large graph datasets. In addition, we make reflections on Neo4j and find out a defect in its result size estimation for aggregation queries.    

There are a number of directions for future work. First, as reflected in experiments in Section \ref{Results and Discussion}, performance of our system is affected by system settings, hence investigation on appropriate system settings would be a valuable follow-up to our work. Second, the system we implemented so far supports SPARQL like queries over schema graph, instead of data graph. It is possible to execute SPARQL-like queries over data graph within our proposed solution framework. The key issue is to take isomorphism into consideration during query decomposition and composition. Finally, greedy selection framework is not a perfect solution for ``Materialization Selection'' problem. More sophisticated solutions can be developed for this hard but valuable problem.


