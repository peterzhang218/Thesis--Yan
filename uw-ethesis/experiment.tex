
%----------------------------------------------------------------------
\section{Experiment Setup}
%----------------------------------------------------------------------

Our main focus is to evaluate different strategies for preprocessing and query evaluation. For instance, the threshold of “Is hot-structure” part in the diagram,  selection policy for materialized substructures in “Cube-Planner” and “Structure-Planner” , different heuristics when ranking sub-structures during decomposition in “Decomposition and Joining” etc.

%----------------------------------------------------------------------
\subsection{Datasets}
%----------------------------------------------------------------------

Big StackOverFlow dataset (44.8GB, with 10 different labels on nodes and 12 different types of edges).

Small StackExchange dataset (2.57GB, same schema with Big StackOverFlow dataset).


%----------------------------------------------------------------------
\subsection{Query Workloads}
%----------------------------------------------------------------------
48 human-readable meaningful queries are written as a query pool. 24 queries are randomly selected as previous workload while the rest 24 are  future workloads. Queries are listed here:

\\ \textbf{Previous WorkLoad:}

User-Comment, Comment-Post: User-UpVotes, Comment-Score, (AVG)Post-Score, Post-PostTypeId=1

User-Comment, Comment-Post: User-Age, (AVG)Comment-Score, Post-PostTypeId=2

User-Comment, Comment-Post: User-ActiveMonth, Post-CreationDate\_Year=2016

User-Comment, Comment-Post: (AVG)User-ActiveMonth, Post-CreationDate\_Year

Badge-User, User-Post, Post-Tag: Tag-TagName, Badge-Date\_Year=2016, Post-CreationDate\_Year

Badge-User, User-Post, Post-Tag: Tag-TagName, Badge-Class

Badge-User, User-Post, Post-Tag: Tag-TagName, Badge-Name=Student

User-Post, Post-Vote: User-UpVotes, Vote-VoteTypeId

User-Post, Post-Vote: User-Ages, (AVG)Post-Score, Vote-VoteTypeId=1

User-Post, Post-Vote: User-Views, Post-CreationDate\_Year=2016, Vote-VoteTypeId

Post-PostLink, Post-Tag: Tag-TagName,Post-CreationDate\_Year, 

Post-PostTypeId, PostLink-LinkTypeId=3

Post-PostLink, Post-Tag: Tag-TagName, Post-CreationDate\_Year

Post-PostLink, Post-Tag: Tag-TagName=database, Post-PostTypeId

Badge-User, User-Post:Badge-Name,Post-Score,Post-PostTypeId=2

Badge-User, User-Post:Badge-Name,(AVG)Post-ActiveMonth,Post-PostTypeId=1

Badge-User, User-Post:Badge-Class, Post-CreationDate\_Year

User-Post, Post-Tag: (AVG)User-CreationDate\_Year, Tag-TagName

User-Post, Post-Tag: User-CreationDate\_Year, (AVG)Post-Score,Tag-TagName

User-Post, Post-Tag: User-Views, (AVG)Post-Score,Tag-TagName

Badge-User: Badge-Name,Badge-Class, Badge-Date\_Year

Post-Tag: Post-CreationDate\_Year, Tag-TagName

Post-Tag: Post-CreationDate\_Year, Tag-TagName

User-Post, Post-PostHistory: User-UpVotes, PostHistory-PostHistoryTypeId

User-Post, Post-PostHistory: User-Age, PostHistory-PostHistoryTypeId=5

Badge-User, User-Comment: Badge-Class, (AVG)Comment-Score



Badge-User, User-Post:(AVG)Post-Score,Post-PostTypeId=2

User-Post, Post-Tag:User-CreationDate\_Year=2016, Tag-TagName

Badge-User, User-Post, Post-Tag: Tag-TagName, Badge-Date\_Year=2016

User-Post, Post-Vote: User-Ages, (AVG)Post-Score, Vote-VoteTypeId=2

Post-PostLink, Post-Tag: Tag-TagName, PostLink-LinkTypeId=3

User-Post, Post-PostHistory: User-DownVotes, PostHistory-PostHistoryTypeId

Badge-User, User-Comment: Badge-Name, (AVG)Comment-Score


\\\textbf{Future WorkLoad:}

Badge-User, User-Post, Post-Tag: Tag-TagName, Badge-Name

User-Post, Post-Vote: User-Views, Vote-VoteTypeId=1

Post-PostLink, Post-Tag: Tag-TagName, Post-PostTypeId=2, PostLink-LinkTypeId

Post-PostLink, Post-Tag: Tag-TagName, (AVG)Post-Score, PostLink-LinkTypeId=1

Post-PostLink, Post-Tag: Tag-TagName, PostLink-LinkTypeId=1

Badge-User, User-Post:Badge-Name, (AVG)Post-Score, Post-PostTypeId

Badge-User, User-Post:(AVG)Badge-Class, Post-CreationDate\_Year=2016

Badge-User, User-Post:Badge-Class,(AVG)Post-Score, Post-PostTypeId

User-Post, Post-Tag: User-Age, (AVG)Post-Score,Tag-TagName

User-Post, Post-Tag: User-Views,Post-Score,Tag-TagName



User-Post, Post-PostHistory: User-Age, PostHistory-PostHistoryTypeId

Badge-User, User-Comment: Badge-Class,Comment-Score



Badge-User: Badge-Class, (AVG)User-ActiveMonth, (AVG)User-Age

Post-Tag: (SUM)Post-ActiveMonth, (AVG)Post-Score, Tag-TagName

User-Comment, Comment-Post: User-UpVotes, Comment-Score, (AVG)Post-Score, Post-PostTypeId=2

User-Comment, Comment-Post: User-UpVotes, (AVG)Post-Score, Post-PostTypeId

User-Comment, Comment-Post: User-Age, Post-PostTypeId

User-Comment, Comment-Post: (AVG)User-ActiveMonth, Post-CreationDate\_Year=2015


%----------------------------------------------------------------------
\subsection{System Setting}
%----------------------------------------------------------------------

We ran the experiments on a Linux cluster machine with 256 GB of memory size.  

Our system is implemented in Java. 

Initial Java vitual machine memory: 100 GB

Maximum Java vitual machine memory: 200 GB
%----------------------------------------------------------------------
\subsection{Neo4j Configuration}
%----------------------------------------------------------------------
Neo4j v4.1.2. 

Initial memeroy size: 60GB. 

Initial memeroy size: 200GB.

We imported Neo4j's official BOLT driver to interact with Neo4j server. The transport protocol is BOLT protocol(a binary protocal supported by Neo4j).

%----------------------------------------------------------------------
\section{Aspects of Interest}
%----------------------------------------------------------------------
\textbf{Patial Materialization}

-  Frequency threshold for “hot structures”.

-  Memory limit.

-  Selection policy for materialized substructures.
 
- Comparison with Jiawei Han’s algorithm on selecting cuboids.

- Comparison with frequent pattern mining algorithm(FPM) on selecting which sub-structures to pre-compute.

\textbf{Future Query Processing}

-  Different heuristics when ranking sub-structures during decomposition (#edges of sub-structure, Score when selected by Structure-Planner, #tuples in the table).

-  Different ways of Decomposation\_Join(“Normal Materializion”, “Informative Materializion”, “Decisive Materializaion”, “Hard Disk Materializaiton”).

Dataset Size

-  Dataset of different sizes.



%----------------------------------------------------------------------
\section{Efficiency Test}
%----------------------------------------------------------------------

%----------------------------------------------------------------------
\subsection{Neo4j BaseLine}
%----------------------------------------------------------------------

%----------------------------------------------------------------------
\subsection{My System}
%----------------------------------------------------------------------
\textbf{Precomputation:}

-  Frequency threshold for “hot structures”. $\leftarrow$ 5

-  Memory limit. $\leftarrow$20GB

-  Selection algorithm. $\leftarrow$ My algorithm

\textbf{Decomposition:}

-  Different heuristics when ranking sub-structures during decomposition. $\leftarrow$ #edges of sub-structure

-  Different ways of Decomposation\_Join $\leftarrow$ “Normal Materializion”

%----------------------------------------------------------------------
\subsection{Frequency Threshold}
%----------------------------------------------------------------------

%----------------------------------------------------------------------
\subsection{Memory Limit}
%----------------------------------------------------------------------

%----------------------------------------------------------------------
\subsection{Selection Algorithms}
%----------------------------------------------------------------------

%----------------------------------------------------------------------
\subsection{View Selection}
%----------------------------------------------------------------------

%----------------------------------------------------------------------
\subsection{Decompose\_Join}
%----------------------------------------------------------------------

%----------------------------------------------------------------------
\section{Discussion}
%----------------------------------------------------------------------
